\begin{center}
    {\LARGE\bf Summary}
\end{center}

Many problems require or benefit from multi-dimensional search, including database querying, computer vision and visualisation. In multi-dimensional search, individual data items are represented as points, vectors or regions in $n$-dimensional space. These are arranged in an indexing structure so relevant data can be retrieved quickly, even when the volume of data is huge.

Different index structures are suited to different kinds of data. This report documents the findings of a project which explores the efficiency of several index structures at processing dynamic, high-dimensional data resulting from scientific, physical simulations. An initial hypothesis is made regarding the expected performance of the structures, which is shown to be incorrect through successive performance evaluations.

An in-depth analysis on the properties of one of the test datasets and scientific data as a whole is performed, concluding the project with a conjecture and multiple hypotheses related to the performance of different classes of index structures when processing scientific data.