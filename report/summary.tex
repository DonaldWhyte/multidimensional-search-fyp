\begin{center}
    {\LARGE\bf Summary}
\end{center}

Many fields require or benefit from multi-dimensional search, including database querying, machine learning, computer vision, and visualisation. In multi-dimensional search, individual data items are represented as points, vectors or regions in $n$-dimensional space. These are arranged in an index structure so relevant data can be retrieved quickly, even when the volume of data is huge.

Different index structures are suited to different tasks. This report documents the findings of a project which explores the point query efficiency of several index structures for dynamic, high-dimensional datasets generated from scientific computation. Particular focus is given to two index structures: the pyramid tree and the point $kd$-tree.

Based on empirical findings, the project concludes with several hypotheses related to the performance of different classes of index structures when processing highly skewed datasets. Specifically, the pyramid tree and most dimension reduction or hash-based index structures are a poor fit for dynamic scientific datasets and other highly skewed data. It is hypothesised that tree-based index structures with adaptive decompositions, such as the point $kd$-tree, are better suited for point queries on these datasets.