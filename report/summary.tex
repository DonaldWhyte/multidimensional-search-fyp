\begin{center}
    {\LARGE\bf Summary}
\end{center}

Many fields require or benefit from multi-dimensional search, including database querying, machine learning, computer vision, and visualisation. In multi-dimensional search, individual data items are represented as points, vectors or regions in $n$-dimensional space. These are arranged in an index structure so relevant data can be retrieved quickly, even when the volume of data is huge.

Different index structures are suited to different tasks. This report documents the findings of a project which explores the point query efficiency of several index structures for dynamic, high-dimensional data generated from scientific computation. An initial hypothesis is made regarding the performance of two index structures, the Pyramid Tree and the point $kd$-tree, which is shown to be incorrect through an evaluation of the structure's performance on real scientific datasets.

An in-depth analysis on the properties of a scientific dataset is then performed, resulting in a conjecture which characterises most scientific datasets as having highly skewed or clustered point distributions.

The project is concluded with several hypotheses related to the performance of different classes of index structures when processing highly skewed datasets. Specifically, the Pyramid Tree and most dimension reduction/hash-based index structures are considered to be a poor fit for highly skewed dynamic data. It is hypothesed that tree-based index structures with adaptive decompositions, such as the point $kd$-tree, are better suited to point queries on highly skewed data. These hypotheses are supported by the empirical findings of the project.