\section{Iteration Summary}

As discussed in Section \ref{sec:iterative-d-and-i}, the Design and Implementation phase of the project was performed in a series of iterations. Each iteration contained some amount of design, implementation and evaluation. The results of the evaluation would determine what was done in the next iteration.

There were four iterations in total. Iteration 0 involved constructing an evaluation framework for generating evaluation measurements, tables and plots for an arbitrary datasets, operation lists and structures. The two baselines were also implemented. Iteration 1 implemented the Pseudo-Pyramid Tree based on the School's existing implementation, and performed a variety of low-level optimisations on the implementation. Performance of the structure was generally quite good when compared to the baselines, but the structure struggled to provide fast queries on the astrophysics dataset. It was suspected that the hashing function was the cause of this performance drop.

Iteration 2 explored this issue further, by using the same underlying implementation of the Pseudo-Pyramid Tree, but changing the hashing function to the Pyramid Tree and another hashing function. The Pyramid Tree hashing function provided better performance than the Pseudo-Pyramid Tree, but still failed to perform well on the astrophysics dataset, as well as another scientific dataset (hurricane Isabel). The other hashing function was very fast, with all datasets, but due to floating point inaccuracies is not usable for many applications.

Finally, iteration 3 involving implementing some variants of the $kd$-Tree and comparing them to the hash-based approaches from the previous iterations. It was found that the $kd$-tree, which being slightly slower on most datasets, was not impacted by either of the scientific datasets. This contradicts the hypothesis made in Section \ref{sec:main-hypothesis}, where it was expected that the Pyramid Tree would perform well for all high-dimensional datasets and the $kd$-Tree would not.