\chapter{Algorithms and Code Listings}
\label{chap:algs-and-code}
\vspace{-0.75cm}
\centerline{\rule{149mm}{.02in}}
\vspace{0.75cm}

This section contains high-level pseudo-code and low-level code listings of algorithms/techniques whose details were not deemed important enough to be in the main report.

\section{Algorithms}

\begin{algorithm}[H]
	\SetAlgoLined
	\SetKwInOut{Input}{input}\SetKwInOut{Output}{output}
	\SetKwFunction{hashPoint}{hashPoint} \SetKwFunction{hashFloat}{hashFloat}

 	\SetKwProg{funcHashPoint}{Algorithm}{}{}
  	\funcHashPoint{\hashPoint{$d, p$}} {
		\Input{$d$ = number of dimensions}
		\Input{$p_0, p_1, \dots, p_{d-1}$ = coordinates of point}
		\Output{$seed$ = integer representing hashed point}
		\Begin {
			$seed = 0$\;
			\For{$i = 0$ to $d - 1$} {
				$seed = seed \oplus \left(\hashFloat(p_i) + \texttt{0x9e3779b9} + (seed << 6) + (seed >> 2)\right)$
			}
			\KwRet{$seed$}
		}
	}

	\caption{Hashing Multi-Dimensional Point in Bucket Hash Table}
	\label{alg:point-hashing}
\end{algorithm}

\paragraph{\textbf{NOTE:}} \texttt{hashFloat} is a function which hashes an individual 32-bit floating point number and corresponds to the function \texttt{float\_hash\_impl2} in Listing \ref{lst:hash-float-function}.

\newpage

\section{Code Listings}

\begin{lstlisting}[label=lst:hash-float-function, caption=Code to Hash Single 32-bit Floating Point Value (Source code from file \texttt{boost/functional/hash/detail/hash\_float\_generic.hpp} in Boost Library 1.42.0)]
// Copyright 2005-2009 Daniel James.
// Distributed under the Boost Software License, Version 1.0. (See accompanying
// file LICENSE_1_0.txt or copy at http://www.boost.org/LICENSE_1_0.txt)

// A general purpose hash function for non-zero floating point values.

inline void hash_float_combine(std::size_t& seed, std::size_t value)
{
    seed ^= value + (seed<<6) + (seed>>2);
}

template <class T>
inline std::size_t float_hash_impl2(T v)
{
    boost::hash_detail::call_frexp<T> frexp;
    boost::hash_detail::call_ldexp<T> ldexp;

    int exp = 0;

    v = frexp(v, &exp);

    // A postive value is easier to hash, so combine the
    // sign with the exponent and use the absolute value.
    if(v < 0) {
        v = -v;
        exp += limits<T>::max_exponent -
            limits<T>::min_exponent;
    }

    // The result of frexp is always between 0.5 and 1, so its
    // top bit will always be 1. Subtract by 0.5 to remove that.
    v -= T(0.5);
    v = ldexp(v, limits<std::size_t>::digits + 1);
    std::size_t seed = static_cast<std::size_t>(v);
    v -= seed;

    // ceiling(digits(T) * log2(radix(T))/ digits(size_t)) - 1;
    std::size_t const length
        = (limits<T>::digits *
                boost::static_log2<limits<T>::radix>::value - 1)
        / limits<std::size_t>::digits;

    for(std::size_t i = 0; i != length; ++i)
    {
        v = ldexp(v, limits<std::size_t>::digits);
        std::size_t part = static_cast<std::size_t>(v);
        v -= part;
        hash_float_combine(seed, part);
    }

    hash_float_combine(seed, exp);

    return seed;
}
\end{lstlisting}