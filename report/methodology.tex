\chapter{Project Methodology}
\label{chap:methodology}
\vspace{-0.75cm}
\centerline{\rule{149mm}{.02in}}
\vspace{0.75cm}

This section gives information on how the project will be tackled, describing the schedule and changes made to the initial schedule. The tools and technologies used throughout the project will also be discussed.

\section{Schedule}
\label{sec:schedule}

The original schedule was devised on 31/01/14,  near the end of the first week of the project. The schedule was broken down into the individual tasks required to complete the defined milestones. These tasks were then grouped into different \textbf{phases}, which mostly run in a serial fashion. Milestones were associated with these phases, each corresponding to the completion of specific deliverables.

On 20/02/14, this schedule was modified slightly to adapt to tasks taking longer than expected. The final project phases and their milestones are:
\begin{enumerate}
	\item \textbf{Project Definition}
	\begin{itemize}
		\item \textbf{Project Outline Defined} -- outline of project's aims and objects is defined 
	\end{itemize}
	\item \textbf{Literature Review and Data Collection}
	\begin{itemize}	
		\item \textbf{Literature Review Complete} -- a literature review of multi-dimensional search has been produced
		\item \textbf{Test Data Collected} -- real test datasets have been decided and collected
	\end{itemize}
	\item \textbf{Mid-project Presentation and Report}
	\begin{itemize}	
		\item \textbf{Mid-Project Presentation Delivered} -- mid-project presentation delivered to Computational Science and Engineering research group
		\item \textbf{Mid-project Report Finished} --  mid-project report submitted
	\end{itemize}	

	\vspace{60pt}

	\item \textbf{Design and Implementation} -- iterative process that involves designing, implementing and optimising index structures as well as evaluating them to determine which perform the best
	\begin{itemize}
		\item \textbf{Software Deliverables Finished} -- final, optimised implementations of the index structures have been produced, ready for a final evaluation
	\end{itemize}
	\item \textbf{Final Report Write-up}
	\begin{itemize}
		\item \textbf{Final Report Finished} -- final report has been submitted
	\end{itemize}
	\item \textbf{Student Symposium}
	\begin{itemize}
		\item \textbf{Final Presentation Delivered} -- final project presentation has been delivered 
	\end{itemize}
\end{enumerate}

Figure \ref{fig:revised-schedule} in Appendix \ref{chap:supp-material} shows a Gantt chart of the project phases, marking the start and end dates for each phase. Notice how some phases are running in parallel, with one phase being the primary focus and the other being a secondary focus where less time is spent. This is a research project by nature, so the exact tasks for most phases were not known in advance. Therefore, it was decided that a more detailed breakdown of the tasks throughout the project would not be created.

Figure \ref{fig:revised-milestone-timeline} shows a timeline marked with every milestone. The location of a milestone on the timeline marks the latest completion date of that milestone.

\subsection{Iterative Design and Implementation}
\label{sec:iterative-d-and-i}

The performance tests and evaluation of the implementations could produce unexpected results, showing that it may be beneficial to implement an entirely different index structure or use a different technology. Therefore, deciding on a \textit{fixed} set of index structures to implement before starting implementation would be unwise. With such an approach, there is no way to go backwards and revise the project plan if performance tests reveal an inefficient index structure.

A decision was made to choose a \textit{single} index structure to implement before starting the Design and Implementation phase. This will be evaluated and based on that evaluation, it will be optimised or it will be discarded in favour of another index structure. This will be repeated in iterations, where each iteration contains the following sub-phases:
\begin{enumerate}
	\item \textbf{Design} -- plan what will be implemented in iteration and design implementation
	\item \textbf{Build} -- implement an index structure or perform a set of optimisations
	\item \textbf{Test} -- perform correctness tests on index structure to ensure it (still) works
	\item \textbf{Performance Analysis} -- perform performance analysis on index structure developed/optimised
	\item \textbf{Evaluation} -- evaluate the results of the analysis and use it to decide what to implement or optimise in the next iteration
\end{enumerate}

Each iteration is a week long, which starts after with the weekly meetings with the project supervisor. This way, the supervisor gets a full summary of the last iteration and can give feedback on the plan for the next iteration. The agendas of each supervisor meeting is provided in Appendix \ref{chap:project-documentation}.

When enough iterations have passed or there is no time left to spend on the Design and Implementation phase, the iterations will stop, with the evaluation of the optimised structure(s) in last phase being the final evaluation that is used in the Final Report Write-up phase. This iterative approach is illustrated in the full project process diagram shown in Figure \ref{fig:full-project-process} in Appendix \ref{chap:supp-material}.

\subsection{Modifications to Original Plan}

The plan has changed since its initial conception. The original plan stated that the initial, unoptimised implementations of the index structures should be finished before the mid-project presentation/report. By finishing the initial implementations of index structures before the presentation, an initial evaluation of the index structures could be performed. The results from this evaluation could be used in the presentation/report to justify decisions on what the remaining project will consist of.

However, due to the scope of the research field, the literature review took one more week than expected. This meant there was little time between the end of the literature review and the start of the mid-project presentation/report. Additional time was required to become fully informed about the field of multi-dimensional search and the nature of the data that will be used for evaluation. This knowledge informs the decisions on which index structures to implement, so it was felt that rushing into an initial implementation may result in a poor decision on which index structures to implement.

It was decided to delay any implementation until after the submission of the mid-project report. By doing this, there was more time to consolidate the research findings and make a more informed decision on the index structures to implement.

For completeness, the Gantt chart and milestone timeline for the original plan is shown in Figures \ref{fig:initial-schedule} and \ref{fig:initial-milestone-timeline} in Appendix \ref{chap:supp-material}.

\subsection{Splay Quadtree}

The first index structure to implement was originally going to be the Splay Quadtree (Section \ref{sec:splay-quadtree}). Due to the complexity of the structure, it was felt that there was not enough time to implement the structure and produce a comprehensive evaluation. It was abandoned in favour of the Pyramid Tree.

\section{Technology}

This section compares some of the potential technologies to use when implementing the index structures and analyse their performance. Justifications are given for the chosen technologies, referring to the project's goals and the experience of the project developer.

\subsection{Programming Language}

There exist many programming languages, all developed for specific purposes. Three programming languages have been considered for this project. This section will describe these languages, discuss their differences and state which will be used for this project and why.

\textbf{Python} is a high-level interpreted language built for general-purpose computing, which has a large standard library and many third-party libraries \cite{python}. \textbf{C} is a low-level systems programming language commonly used for high-performance applications \cite{c-lang}. \textbf{C++} is a middle-level, compiled systems programming language, combining low-level features (e.g. manual memory management) and high-level features (e.g. object-orientated programming) \cite{cpp}. C/C++ applications are compiled straight to the native machine's CPU instruction set, meaning less time is spent translating the code to instructions the hardware can directly execute. \textbf{Haskell} is a purely functional general-purpose programming language which, like C/C++, compiles to native CPU instructions \cite{haskell}.

The following aspects have been considered when deciding which language to use:
\begin{itemize}
	\item \textbf{Performance} -- how fast the final application can run. This is based on a number of factors, such as how many intermediary layers there are between the code and the physical machine. Higher-level languages require the computer to do more work, as it has to convert the application's instructions into CPU instructions that are executed directly on the processor.
	\item \textbf{Ease/Speed of Development} -- this encompasses multiple aspects of a language. To facilitate fast development, the language must have a mature tool set, good documentation and support and be easy to understand.
\end{itemize}
 
Executing a Python statement introduces more overhead than C/C++ or Haskell statements. This is because Python is a high-level, interpreted language, where statements are compiled into bytecode at runtime which is then run on a virtual machine, which translates the bytecode to native assembly instructions. Therefore, many Python applications run slower than C/C++ or Haskell applications, which are pre-compiled to native CPU instructions. However, due to the large amount of experience the developer has in Python, its large standard library and the higher number of abstractions, it is likely Python would result in the fastest development time. 

Haskell's purely functional nature makes it fundamentally different than most popular languages and the developer of this project also has little experience writing programs in a purely functional style. It is predicted this will increase the time it takes to implement the structures. Additionally, it is unlikely well-written, optimised Haskell code will be produced in such a short time. If parallel index structures are implemented, however, the lack of state and built-in support for parallelisation \cite{parallel-haskell} would make Haskell a powerful choice.

Despite development speed being important, the primary aim of the project is the acceleration of search. Haskell can be used to write highly optimised code, especially when implementing parallel algorithms to utilise multiple CPU cores. C++ gives the developer more control over the computation and management of memory, allowing for low-level performance tuning. The trade-off for this control is the added complexity of the language \cite{cpp-hard}, leading to an increased amount of code to write and the decrease in development speed that follows.

There is much debate on whether C produces faster code than C++ (e.g. \cite{c-vs-cpp1, c-vs-cpp2, c-vs-cpp3}). This report will not make assertions about the relative performance of these two languages, but if C \textit{is} generally faster than C++ and the primary goal of the project is speed, then it seems logical to choose C.

In a project such as this, where many ideas will be implemented, tested and potentially thrown away, programmer productivity is still vital. The importance of this is amplified by the project's short duration. While the developer has experience using C, they are still significantly more experienced with C++. Therefore, as a compromise between the goal of high performance and the programmer's current skill set, C++ has been chosen as the language to use for developing the initial implementation of the index structure. 

\subsection{Development Tools}
\label{sec:development-tools}

Various tools will be used throughout the development of the structures. \textbf{CMake} \cite{cmake} will be used to automate the C++ build process. Using a build automation tool allows one to write high-level build scripts that are cross-platform, increasing the portability of the implemented index structures.

Unit tests will be written to ensure the implemented index structures and all the associated algorithms are functioning correctly. A unit test ensures a single unit of the code is exhibiting the desired behaviour. A test is performed in isolation, preventing other code in the application from being executed so it does interfere with the test's results \cite{automated-defect-prevention}. A unit test framework makes the process of writing unit tests easier and faster \cite{unit-test-frameworks}. C++ has many unit testing frameworks, such as CppUnit \cite{cppunit}, NullUnit \cite{nullunit} and Google Test \cite{google-test}. \textbf{Google Test} has been chosen for this project because of its wide feature set and comprehensive documentation, which is available at \cite{google-test}.

Version control is a way of tracking changes to textual documents, as well as \textit{who} made the changes \cite{pragmatic-version-control}. A VCS (version control system) will be used to track the changes of the implementation's source code, tests and built automation scripts. Despite this project only having a single developer, there are multiple reasons it has been decided to manage the source code using a VCS. Firstly, it allows the developer to easily keep a log of the changes made to any file and \textit{why} those changes were made. One can also revert source files to prior versions if bugs are found or previously deleted code is required again, Finally, using a VCS means there will be multiple copies of the source code, which can be used as backups to mitigate the damage caused by data loss. The distributed VCS \textbf{Git} \cite{git} will be used for this project, due to the developer having prior experience in the tool, meaning less time is taken from implementation to learn a new tool.

\subsection{Performance Analysis Tools}

In addition to considering the theoretical performance of the chosen algorithms, the implementations' performance will be tested using profiling. Profiling is a way of measuring the performance of a program or system \cite{efficient-cpp} and is used to provide insight into which parts of the code take the longest or use the most memory (i.e. where the performance bottlenecks are). A profiler is a tool which measures some performance metrics of a program. Since the core goal of this project is accelerating multi-dimensional search, being able to measure the performance of the implementations and identify where the performance bottlenecks are is incredibly useful.

To optimise the implementations, a profiler that can measure heap usage and cache misses will be used. One which can produce call graphs with timings, that measure the flow of execution and how much time is spent in each function, is also desired. Using a profiler that produces little slow down is useful, but not critical, since timings could be scaled to take the extra overhead into consideration. 

Many profilers were considered when choosing one for this project. Intel VTune \cite{intel-vtune} is feature-rich profiler for Intel-based machines, but it is proprietary. Valgrind \cite{valgrind}, Profiny \cite{profiny} and gperftools \cite{gperftools} are freely available profilers. Out of these, only Valgrind can measure both heap usage and cache misses. gperftools can generate visualisations of program flow and where the program spends most of its time using call graph images. gperftools provides a similar feature for heap profiling as well. Valgrind and gperftools will both be used to measure cache misses and generate call graphs images.

\subsection{SSE Optimisation}
\label{sec:sse}

SIMD stands for Single Instruction, Multiple Data and was defined by Flynn as a classification of parallel computing \cite{flynns-taxonomy}. In SIMD, a single instruction is applied to multiple data items at the same time. If $p$ is the maximum number of data items that can be operated on in parallel at a time, then ideally a computation can be made $p$ times faster. However, it is only suitable for computations where the same operation can be applied to multiple data items independently, and the order in which those operations complete does not affect the final output.

Streaming SIMD Extensions, or SSE, is a specification of an instruction set that performs SIMD operations on the widely used x86 CPU architecture \cite{sse}, which is the architecture used for the development and test environment of this project. SSE will be used to accelerate frequently executed code where possible.

\section{Conclusion}

The project schedule follows a waterfall approach, with the end of one phase leading to the start of the next. To take unexpected change into account, flexibility is required. For this reason, one of these phases is composed of several weekly iterations.

C++ will be used to implement all index structures in the project. Both gperftools and Valgrind will be used to profile the implementations.