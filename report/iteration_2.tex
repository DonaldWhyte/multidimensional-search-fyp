\section{Iteration \#2 -- Pyramid Tree and Other Hash-Based Approaches}

This iteration explores other hashing functions to find a hash-based index structure which performs well on the two scientific datasets.

\subsection{Pyramid Tree}

The Pyramid Tree implementation is identical to the Pseudo-Pyramid Tree implementation, except points are now hashed to their pyramid value as described in Section \ref{sec:pyramid-tree-detail}.

\subsection{B${}^+$-Tree as Underlying Structure}

Berchtold et al. originally used a B${}^{+}$-tree when developing the Pyramid Tree \cite{pyramid-tree}. It was decided that using the same underlying search structure would allow for a more fair comparison of the Pyramid tree data structure. Two B${}^{+}$-tree implementations, bpt\cite{bpt} and cpp-btree\cite{cpp-btree}, were used as the underlying structure for the Pyramid Tree. The first is a C++ implementation while the second is a pure C implementation.

Table \ref{tab:hashtable-bplus-time-comparison} provides total execution times of each operation, measured using Insert-Query-Delete operation list with 500,000 points from the 16D uniform synthetic dataset. There is a substantial difference between the speed of the hash table Pyramid tree and the B${}^{+}$-tree implementations. After profiling, it was discovered the main cause of the decrease in speed was simply the additional overhead incurred by splitting and merging nodes in the B${}^{+}$-tree. This matches the theoretical performance analyses of the two structures, where it is shown that hash tables and B${}^{+}$-trees have amortised $O(1)$ and $O(\log n)$ operations respectively (the former being dependant on the hashing function used).

Based on these results, it has been decided to continue using the hash table, not the B${}^{+}$-tree, as the underlying search structure.

\begin{table}
	\centering
	\begin{tabular}{|l|l|l|l|}
		\hline
		\textbf{Operation} & \texttt{boost::unordered\_map} & bpt & cpp-btree  \\
		\hline
		\textbf{Delete} & 0.211423 & 0.450339 & 0.975521 \\
		\textbf{Insert} & 0.379671 & 0.680246 & 1.278914 \\
		\textbf{Point Query} & 0.177153 & 0.259851 & 0.548411 \\
		\hline
	\end{tabular}
	\caption{Total Execution Time (in Seconds) of Pseudo-Pyramid Tree With Different Underlying 1D Index Structures (16D Randomly Uniform Dataset, 500,000 Points)}
	\label{tab:hashtable-bplus-time-comparison}
\end{table}

\subsection{Bucket Hash Table}

% http://stackoverflow.com/questions/7403210/hashing-floating-point-values
% https://svn.boost.org/trac/boost/ticket/4038
% http://programmers.stackexchange.com/questions/63595/0x9e3779b9-golden-number
% http://stackoverflow.com/questions/4948780/magic-number-in-boosthash-combine
% http://burtleburtle.net/bob/hash/doobs.html

If larger bucket sizes mean more point comparisons are performed on average, then it is not unreasonable to expect an average bucket size of 1 to provide very good performance. After all, if the hashing function is an $O(d)$ operation, then it follows that \texttt{insert}, \texttt{delete} and point query operations take $O(d)$ time. 

The Bucket Hash Table also uses a \texttt{boost::unordered\_map} in the back-end, but instead of trying to exploit the spatial properties of points directly to provide good bucket size, a different, more general-purpose hashing function provided in the Boost library is used. Like the other hash-based structures discussed in this section, it is possible for two points to be hashed to the same value. However, from empirical performance tests, it has been shown that the average bucket size is almost always near one (bucket size measurements are provided in the next section), meaning most operations are performed in $O(d)$ time.

A high-level algorithm describing the hashing function is given in Algorithm \ref{alg:point-hashing} in Appendix \ref{chap:supp-material}. This function hashes each individual coordinate (floating point value) and combines them using exclusive-or ($\oplus$) and bitshifting operations. A magic number representing the reciprocal of the golden ratio, $\phi = \frac{1 + \sqrt{5}}{2}$, is used when combining the hash values of individual coordinates. The choice to use the golden ratio was inspired by Jenkins' hash function\cite{hash-combine}, where it is used to ensure consecutive floating point values will be mapped to integers with large distances between them. This increases the likelihood of having points distributed more uniformly across buckets when points are clustered within a small numerical range (like they are in the astrophysics dataset).

One major issue with this approach is the potential for floating-point inaccuracy to give incorrect results. On the controlled performance tests executed in this project, a point is queried using the exact same floating point values for the coordinates as when it was inserted. This means the two points have the same identical bit patterns. 

Suppose the point to query was the output of a more complex computation, where rounding errors may come into play. Even if an equal point conceptually is being stored in the structure, rounding errors may cause the two points to have different bit patterns, potentially resulting in different hashed values. The output point, while being stored in the structure, will appear as if it is not. This is a common issue when using floating point values in hashing functions.

Therefore, the Bucket Hash Table may be unreliable for certain applications, especially ones where the input points are the result of computations involving many arithmetic operations.

\subsection{Bucket Statistics}

The same underlying implementation is used for the Pseudo-Pyramid Tree, Pyramid Tree and Bucket Hash Table; only the hashing function varies. Pyramid Tree's and Bucket Hash Table's hash functions have also been parallelised using SSE, which has made the difference between execution times between the hash functions of the three structures very small. The core factor which determines the performance of each structure is still bucket size. Therefore, statistics on the bucket size will be used to determine which structure is superior. As in iteration one, one synthetic dataset and all three real datasets will be used for bucket size measurement.

\begin{table}
	\centering
	\makebox[\textwidth][c]{%
		\begin{tabular}{|l|l|l|l|l|}
			\hline
			& & \multicolumn{3}{c|}{\textbf{Bucket Size Statistics}} \\
			\hline
			\textbf{Dataset} & \textbf{Time to Query (sec)} & \textbf{Average} & \textbf{St. Dev} & \textbf{Max} \\
			\hline
			\textbf{Pseudo-Pyramid Tree} & & & & \\
			500,000 16D Random Points & 0.105091 & 1.0312 & 0.115177 & 4 \\
			500,000 Astrophysics Points & 70.4391 & 3586.57 & 24,528 & 235260 \\
			500,000 Hurricane Isabel Points & 69.3996 & 17323.66 & 83,533 & 293949 \\
			435,544 3D Armadillo Mesh Points & 0.141219 & 19.1465 & 14.9386 & 187 \\
			\hline
			\textbf{Pyramid Tree} & & & & \\
			500,000 16D Random Points & 0.177153 & 1.16285 & 0.444827 & 7 \\
			500,000 Astrophysics Points & 60.0216 & 358645.7 & 54145.2 & 143496 \\
			500,000 Hurricane Isabel Points & 69.4891 & 248977 & 248976 & 497953 \\
			435,544 3D Armadillo Mesh Points & 0.13448 & 2.45933 & 24.3335 & 1173 \\
			\hline
			\textbf{Bucket Hash Table} & & & & \\
			500,000 16D Random Points & 0.249102 & 1.01004 & 0.100054 & 3 \\
			500,000 Astrophysics Points & 0.172516 & 1.01057 & 0.104108 & 4 \\
			500,000 Hurricane Isabel Points & 0.222288 & 1.00987 & 0.0993287 & 3 \\
			435,544 3D Armadillo Mesh Points & 0.0811833 & 1.00857 & 0.0928895 & 3 \\
			\hline
		\end{tabular}
	}%
	\caption{Statistics on Bucket Size, Based on Dataset, of All Hash-Based Structures}
	\label{tab:perf2-bucket-stats}
\end{table}

Table \ref{tab:perf2-bucket-stats} contains statistics on bucket size of all three hashing functions, using all real datasets and the 16D synthetic dataset.

 TODO: discussion of results and implications

\subsection{Summary}

TODO
