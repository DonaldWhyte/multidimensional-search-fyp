\chapter{Project Conclusion}
\label{chap:project-conclusion}
\vspace{-0.75cm}
\centerline{\rule{149mm}{.02in}}
\vspace{0.75cm}

This section will discuss the limitations of the project's work, future work to support or extend the project's findings and discusses the project's success.

\section{Limitations}

TODO: never had time to verify new hypothesis! beyond scope to verify them or conjecture!!

This project has only worked with three synthetic datasets and three real datasets. Only two of these datasets represent the kind of data this project is targeting, which is data from scientific computation. Using more data from a scientific domain and performing further analysis on their properties may have derived more insight into which index structures are suited to such data.

A large chunk of the Design and Implementation phase was spent focusing on the Pseudo-Pyramid tree and Pyramid Tree before they were found to perform poorly with point queries on the scientific datasets. By the time the limitations of these structures were discovered, there was not enough time to go into as much depth with the implementation or evaluation of other structures. Originally, it was intended that the bucket $kd$-tree (see Appendix \ref{chap:additional-structures}) would be discussed and evaluated alongside the point $kd$-tree, but the remaining time did not allow for this.

\section{Future Work}

The end of this project resulted in a conjecture and three hypotheses being made, meaning potential future work includes exploring these in much greater detail. This section suggests some of the ways these hypothese	s can be explored further.

To verify Hypothesis 1, the Pyramid Tree's suitability with higher-dimensional scientific datasets can be examined in further detail by testing it with several more scientific datasets. Non-scientific datasets can even be used to determine the types of data the structure can process efficiently. Examining a larger number of scientific datasets could be used to further explore the conjecture that most scientific datasets have dense clusters of points in data space and large sparse regions. In fact, studying the mathematical challenges of $\mathbb{R}^m \rightarrow \mathbb{R}^d$, where $d > m$, is a whole project in itself.

Exploring Hypothesis 2 would involve implementing and evaluating several adaptive dimension reduction or hashing techniques, perhaps using the same test datasets as this project. Examples of techniques to research include iDistance \cite{idistance} and the Multigrid proposed in Appendix \ref{chap:additional-structures}.

The most important insight in the project is the superior performance of a tree-based index structure for the target data. Verifying Hypothesis 3, which claims most dynamic tree-based structures are more suitable to scientific datasets than dimension reduction techniques, would involve implement several dynamic tree-based structures. Again, these could be evaluated sing the same test datasets as this project. Examples of such structures include bucket KD-trees and BD-trees \cite{kdtree-v-bdtree}, as well as the KDB-tree, Quadtreap and the Splay Quadtree structures mentioned in Sections \ref{sec:recursive-partition-structures} and \ref{sec:history-sensitive-structures}.

Finally, embedding, distance-based or parallel structures (all briefly mentioned in Section \ref{chap:background_research}) are avenues to explore. Based on the literature review, little research has been done on the suitability of these structure types for point queries on dynamic data.

\section{Evaluation of Methodology}

The project methodology refers to how tasks were organised and approached. The majority of the project's schedule was sequential, having one major task following another. The Design and Implementation phase however was iterative. There were multiple iterations of the same sub-process, which encompassed design, implementation and evaluation. 

It is felt that this hybrid approach worked well. At the start of the project, the author had little knowledge about multi-dimensional search. Just to determine the scope and objectives of the project, it required several weeks of background research. Synthesis this research into a mid-project report and mid-project presentation took a significant amount of time, and each task was dependent on the tasks before. Therefore, it makes sense to have this initial part of the project serial.

Implementing and testing index structures was the most exploratory part of the project, so flexibility was required. Making the Design and Implementation phase worked well because it allowed the project to be flexible. At the end of each iteration, the evaluation allowed one to gather insight into the implemented structures' performance.  During the first iteration, the Splay Quadtree was abandoned due to the significant amount of effort required to implement, so it was discarded in favour of the Pyramid Tree. After the first two iterations, it was found that the Pyramid Tree had poor performance with the target datasets, so the next iteration turned to entirely different kind of structure. Planning exactly what should be implemented in advance would be a waste, since new findings would have quickly became irrelevant.

One aspect of the schedule that could have been improved was the amount of time left for writing the final report. While the schedule left four weeks to write the report, there were multiple unexpected findings has been found. These results required more time to evaluate and discuss, increasing the time it took to write the final evaluation. Perhaps the schedule should have left another week as a buffer in the case of unexpected findings, which are to be expected in a research project. That being said, it was felt the methodology suited the project well.

\section{Objectives and Minimum Requirements}

Sections \ref{sec:objectives} and \ref{sec:requirements} lists the project's objectives and minimum requirements respectively. The minimum requirements were developed so that meeting them also meets the project's objectives. For brevity, the minimum requirements will be referred to by their numbers.

Requirement 1 was met by producing a literature review which has been placed in this report. The challenges of multi-dimensional search were summarised, most classes of index structures were discussed in some way and a taxonomy showing how index structures have affected each other was developed. The literature review itself was comprehensive

Requirement 2 was met by implementing the initial Pseudo-Pyramid Tree implementation, which was analysed at the end of Section \ref{sec:iteration1}. This requirement was largely exceeded, since a total of twelve implementations were developed for this project. Eight of these implementations being discussed and analysed with respect to both performance timings and their general behaviour (e.g. bucket size and balance factor). Due to time constraints, the remaining four structures could not be discussed and evaluated in detail. These are described in Appendix \ref{chap:additional-structures}.

Requirement 3 was met by optimising the Pseudo-Pyramid Tree hashing function both algorithmically and through SSE. Several implementations of the hash-based index structures were developed to reduce cache misses and computation. Several additional structures were also implemented to determine which provides the best performance for the target datasets. Therefore, this requirement was also exceeded.

Evaluation was present throughout the Chapters \ref{chap:design-and-implementation} and \ref{chap:technical-evaluation}, meaning Requirement 4 has been met. Again, the requirement was exceeded because three evaluations based on empirical performance timings were performed, along with a separate evaluation of the high-level algorithms deriving insight into what kind of index structures may be suitable for the project's target data.

Therefore, all the minimum requirements were met and exceeded to varying degrees. In particular, the amount of optimisations and index structures that were implemented and evaluation was much larger than intended. This was done to ensure the evaluation was a comprehensive as possible within the three months of the project.

\section{Deliverables}

The deliverables described in Section \ref{sec:deliverables} were all produced in the following ways:
\begin{enumerate}
	\item Source code of the evaluation framework and all implemented structures, along with documentation, has been submitted alongside this report
	\item mdsearch, a C++ library containing implementations of some of the index structures explored in the project, has been released to the public. Where to get mdsearch, as well as documentation describing what features mdsearch provides and how to use them, is available in Appendix \ref{chap:mdsearch}.
	\item An evaluation of the index structures was performed as part of Requirement 4
	\item Generated synthetic data can be re-produced using the data generators provided in the source code of Deliverable 1
\end{enumerate}

\section{Conclusion}

To conclude, this project implemented several index structures and evaluated their performance on dynamic high-dimensional data, specifically focusing on scientific datasets. It was found the Pyramid Tree and a similar structure, the Pseudo-Pyramid Tree, consistently gave poor point query performance with scientific datasets. A well-established structure, the point $kd$-tree outperformed these two structures by a large margin. An evaluation of these structures' behaviour with the astrophysics dataset was performed, in which some conjectures and hypotheses were derived regarding the suitability of different classes of index structures for dynamic scientific datasets. Future work has been proposed to verify these hypotheses and explore the discussed issues in greater detail.

The overall aim of the project was to implement one or more index structures and evaluate their performance with respective to high-dimensional datasets (i.e. $d > 10$). Using the astrophysics dataset as a test case, the evaluation went into much greater detail than initially expected. A greater number of index structures and optimisations were implemented than planning, resulting in a more comprehensive evaluation. Therefore, the project's aim has been met and exceeded.

Furthermore, the evaluation resulted in a conjecture and several hypotheses being made. Doing so has created a foundation for future work to be started. The project has resulted in several index structures being implemented, which have been evaluated in detail. Additionally, a platform for future work has been proposed, As such, I would consider the project a success.