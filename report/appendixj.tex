\chapter{Real Application of Search Structures}
\label{chap:mdsearch-application}
\vspace{-0.75cm}
\centerline{\rule{149mm}{.02in}}
\vspace{0.75cm}

The performance of four index structures from the mdsearch library (see Appendix \ref{chap:mdsearch}) were tested in two scenarios. The first test involves processing the full astrophysics dataset. The second test involves a trace of structure operations from a real application.

\section{Full Astrophysics Dataset}

The full astrophysics dataset contains 36,902,399 points. Table \ref{tab:full-astrophysics-times} shows the execution times of point insertions, deletions and queries on four structures from the mdsearch library. The Insert-Query-Delete operation list was used to measure these times. How these times were measured, and the astrophysics dataset, are described in greater detail in Chapter \ref{chap:evaluation-outline}.

\begin{table}
	\centering
	\begin{tabular}{|l|l|l|l|l|}
		\hline
		\textbf{Operation} & Point $kd$-tree & Pyramid Tree & Multigrid & Bit Hash \\
		\hline
		Delete & 729479.3 & 2996.97 & 80.7868 & 13.6403 \\
		Insert & 24.5095 & 2787.87 & 136.421 & 18.5444 \\
		Point Query & 24.5907 & 2781.16 & 135.492 & 13.9206 \\
		\hline
	\end{tabular}

	\caption{Execution Times (in seconds) of mdsearch Structures with Full Astrophysics Dataset}
	\label{tab:full-astrophysics-times}
\end{table}

The results are similar to those shown in Chapter \ref{chap:performance-evaluation}. Bit Hash outperforms all the other structures. The point $kd$-tree is the next fastest for insertions and queries. Individual point deletion in point $kd$-trees is incredibly slow, showing that the structure is not suitable for individual point deletions. Again, the Pyramid Tree degenerates to a semi-sequential scan, taking thousands of seconds to insert, query or delete all the points. Multigrid's performance is better, taking approximately 135 seconds to query every point. However, like the Pyramid Tree, the highly clustered distributions in the astrophysics dataset still cause Multigrid to degenerate.

\section{Joint Contour Net Construction}

Joint Contour Nets (JCNs) can be used to visualise the contours of multi-variate functions \cite{jcn}. Computing Joint Contour Nets requires a computationally expensive algorithm. An index structure can be used to accelerate this algorithm.

As the algorithm runs, it processes one cell, or local neighbourhood, at a time. There is a parameter $w$ that determines the width of the interval used to slice the multi-variate fields. Smaller $w$ produces more points. A value of 4 has been used for $w$ in these tests.

Two traces of index structure operations were generated from an application that computes Joint Contour Nets. The two traces correspond to two different ways of computing the Joint Contour Net. The \textit{active} trace is generated by inserting and querying points into an index structure and then clearing the structure when it processes the next cell. The \textit{centers} trace does not reset the structure per cell. Each trace contains point insertions, queries and resets (which remove every point).

\begin{table}
	\centering
	\begin{tabular}{|l|l|l|l|l|l|}
		\hline
		\textbf{Trace} & \textbf{\# Operations} & Point $kd$-tree & Pyramid Tree & Multigrid & Bit Hash \\
		\hline
		Active ($w = 4$) & TODO & 0.336738 & 0.709166 & 0.923040 & 1.95543 \\
		Centers ($w = 4$) & TODO & 7.63283 & 14116.3 & 1450.32 & 3.69349 \\
		\hline
	\end{tabular}

	\caption{Execution Times (in seconds) of mdsearch Structures with Joint Contour Net Traces}
	\label{tab:jcn-trace-times}
\end{table}

Table \ref{tab:jcn-trace-times} shows the total execution time for each structure when running these traces. The point $kd$-tree outperforms all the other structures, even Bit Hash, with the \textit{active} trace. The Pyramid Tree comes in a close second, followed by Multigrid. Bit Hash is actually the slowest structure here. This is because Bit Hash allocates much more memory, which induces more overhead. This overhead is neglible as the dataset size increases, but the structures remain fairly small in the \textit{active} trace because they are periodically reset. This is probably why the Pyramid Tree has such good performance -- there are never enough points in the structure to make it degenerate.

With \textit{centers}, Bit Hash again becomes the fastest, because the structures store many more points at a time. Point $kd$-tree comes in second, followed by Multigrid. The Pyramid Tree degenerates to sequential scan, taking over 14,000 seconds.

These results show that point $kd$-trees give the best (reliable) performance when computing Joint Contour Nets, when either method is used to compute them.

\section{Conclusion}

The results from both applications show that the point $kd$-tree provides efficient insertions and queries, even for large, highly skewed datasets. Overall, it is the fastest reliable index structure implemented in the project and in the mdsearch library.