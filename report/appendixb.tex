\chapter{Record of External Materials Used}
\vspace{-0.75cm}
\centerline{\rule{149mm}{.02in}}
\vspace{0.75cm}

\section{Data}

The real datasets used for evaluation in this project were retrieved from the following sources:
\begin{itemize}
	\item \textbf{Astrophysics Turbulence Simulation} -- from the IEEE Visualization 2008 Contest, which released a dataset generated by Whalen and Norman \cite{astrophysics-dataset}
	\item \textbf{Hurricane Isabel Simulation} -- from the IEEE Visualization 2004 Contest, which released a dataset generated by the National Center for Atmospheric Research in the United States \cite{hurricane-isabel-dataset}
	\item \textbf{Armadillo 3D Mesh} -- from the Stanford 3D Scanning Repository \cite{armadillo-mesh}
\end{itemize}

\section{Code}

Aside from the C++ standard library and the Boost library, the only externally written source code was the original Pseudo-Pyramid tree. This was written by Zhao Geng\footnote{Z.Geng@leeds.ac.uk}, who is a research associate within the School of Computing at University of Leeds (at the time of writing).