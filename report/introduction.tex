\chapter{Introduction}
\label{chap:introduction}
\vspace{-0.75cm}
\centerline{\rule{149mm}{.02in}}
\vspace{0.75cm}

The real world is complex and multivariate, with high degrees of freedom. This is reflected in the complexity of the data it produces. Multi-dimensional search can be used to efficiently reason about this complex data.

Many fields require or benefit from multi-dimensional search, including database querying, machine learning, computer vision and visualisation. In multi-dimensional search, individual data items are represented as points, vectors or regions in $n$-dimensional space. These are arranged in an index structure so relevant data can be retrieved quickly, even when the volume of data is huge.

Since multi-dimensional search is a problem that appears in so many areas of computer science, developing fast, efficient ways of building and querying index structures has been the focus of many researchers over the past forty years. Dozens of multi-dimensional search structures have already been developed and there is still much ongoing work to develop even more efficient search structures. 

\section{Aim}

This project will survey existing multi-dimensional search structures that attempt to solve the multi-dimensional search problem efficiently. The major challenges in the field, and how they have guided the development of structures throughout the past four decades, will be identified. The core aim of the project is to implement one or more index structures, optimising them specifically for high-dimensional datasets. These implementations will then be evaluated with respect to their proposed performance in the literature and pre-defined baselines, using chosen test datasets. This evaluation will highlight performance bottlenecks in the implementations and fundamental limitations of the algorithms themselves.

\newpage

\section{Objectives}
\label{sec:objectives}

The objectives of the project are:
\begin{enumerate}
	\item Understand the current state and challenges of multi-dimensional search
	\item Implement multi-dimensional search structures
	\item Perform performance analysis on implementations and attempt to optimise structures for greater performance on high-dimensional data
	\item Evaluate and compare performance of final structures to pre-defined baselines, unoptimised structures and their reported performance in the literature
\end{enumerate}

\section{Requirements}
\label{sec:requirements}

The minimum requirements are closely related to the objectives of the project. They are:
\begin{enumerate}
	\item Produce literature review describing the current state of multi-dimensional search structures, comparing their strengths and weaknesses and highlighting core challenges of the field
	\item Implement at least one index structure and perform performance analysis of the implementation
	\item Perform one set of modifications to the implemented structure in an attempt to optimise its performance
	\item Evaluate and compare performance of final implementations to pre-defined baselines, unoptimised structures and their reported performance in the literature
\end{enumerate}

Possible extensions of the project include:
\begin{enumerate}
	\item Parallelising implementations using technologies such as Haskell, CUDA (GPGPU) or MPI
	\item Developing an entirely new index structure, which is also implemented and evaluated
\end{enumerate}

\section{Deliverables}
\label{sec:deliverables}

The following will be delivered upon project completion:
\begin{enumerate}
	\item Documented source code of implemented index structures
	\item User manual describing how to use the index structures
	\item Evaluation of the performance of the implemented index structures, with respect to pre-specified test data
	\item Generated synthetic data that is used for the evaluation
\end{enumerate}

\newpage

\section{Project Scope}
\label{sec:core-assumptions}

A research group in the School of Computing at the University of Leeds have a particular interest in the applications of multi-dimensional search for scientific visualisation. This project will focus on \textit{scientific visualisation} and not other applications, such as database indexing. Therefore, the performance of the implemented index structures will be evaluated primarily using high-dimensional scientific datasets with continuous domains, instead of discrete data pulled from databases.

Throughout the project, some assumptions have been made. These are used to narrow the scope of the project and allow more time to be spent focusing on the core aim of the project, which is high-dimensional scientific data. The assumptions are:
\begin{enumerate}
	\item Datasets have a ``high" number of dimensions ($\geq 10$), meaning the performance of the index structures will be measured using data with at least 10 dimensions. Data using less dimensions may be used to understand how the implementations behave with respect to dimensionality.
	\item Datasets will be able to fit into the main memory (i.e. RAM) of the machine used for evaluation. That is, none of the data is paged to secondary memory, so page accesses causing reduced performance does not need to be considered.
	\item Datasets are dynamic, meaning points may be inserted, deleted or updated at any time.
	\item Structures only store \textit{unique} points, meaning a structure cannot store two identical points
\end{enumerate}

There are several ways of querying the data stored inside index structures. The relevant research group is interested in \textit{point query} performance, so it was decided point queries will be the focus of the project.

\section{Report Structure}

Chapter \ref{chap:methodology} explains how the project was scheduled and the technologies used to produce the software deliverables. Chapter \ref{chap:background_research} reviews the existing work performed in this field and the core challenges which have driven the development of index structures throughout the past forty years. This leads into Chapter \ref{chap:chosen-structures}, which describes the index structures to be implemented and evaluated in detail, explaining why they were chosen.

Chapter \ref{chap:design-and-implementation} gives details on how the chosen index structures were implemented, documenting the low-level optimisation efforts performed to accelerate the structures. Chapter \ref{chap:evaluation-outline} covers what measurements and data will be used to evaluate the index structures' performance. Chapter \ref{chap:performance-evaluation} gives an evaluation of the implemented index structures using empirical performance timings. 

Chapter \ref{chap:technical-evaluation} discusses the results from performance evaluation in greater detail, exploring the nature of scientific datasets and the suitability of different classes of structures for these types of datasets. Chapter \ref{chap:project-conclusion} concludes the project by proposing future work to explore the hypotheses made in the previous chapter.