\chapter{Introduction}
\label{chap:introduction}
\centerline{\rule{149mm}{.02in}}
\vspace{2cm}

Many problems require or benefit from multi-dimensional search, including database querying, computer vision and visualisation. In multi-dimensional search, individual data items are represented as points, vectors or regions in $n$-dimensional space. These are arranged in an indexing structure so relevant data can be retrieved quickly, even when the volume of data is huge.

Since multi-dimensional search is a problem that appears in so many areas in computer science, developing fast, efficient ways of building and querying indexing structures has been the focus of many researchers over the past forty years. Dozens of multi-dimensional search structures have already been developed and there is still much ongoing work toward developing even more efficient search structures. 

\section{Aim}

This project will survey existing multi-dimensional search structures which attempt to solve the multi-dimensional search problem efficiently, identifying the major challenges of the field and how this has guided the development of structures throughout the last two decades. The core aim of the project is to implement one or more index structures, optimising them specifically for high-dimensional datasets. These implementation(s) will then be evaluated with respect to their proposed performance in the literature and pre-defined baselines, using chosen test datasets. This evaluation will highlight the found performance bottlenecks in the implementations and limitations of the algorithms themselves.

\section{Objectives}
\label{sec:objectives}

The objectives of the project are:
\begin{enumerate}
	\item Understand the current state and challenges of multi-dimensional search
	\item Implement multi-dimensional search structures
	\item Perform performance analysis on implementations and attempt to optimise structures for greater performance on high-dimensional data
	\item Evaluate and compare performance of final structures to pre-defined baselines, unoptimised structures and their reported performance in the literature
\end{enumerate}

\section{Requirements}
\label{sec:requirements}

The minimum requirements are closely related to the objectives of the project. They are:
\begin{enumerate}
	\item Produce literature review describing the current state of multi-dimensional search structures, comparing their strengths and weaknesses and highlighting core challenges of the field
	\item Implement at least one index structure and perform performance analysis of the implementation
	\item Perform one set of modifications to the implemented structure in an attempt to optimise its performance
	\item Evaluate and compare performance of final implementations to pre-defined baselines, unoptimised structures and their reported performance in the literature
\end{enumerate}

\section{Deliverables}
\label{sec:deliverables}

The following will be delivered upon project completion:
\begin{enumerate}
	\item Documented source code of implemented index structures
	\item User manual describing how to use the index structures
	\item Evaluation of the performance of the implemented index structures, with respect to pre-specified test data
	\item Generated synthetic data that is used for the evaluation
\end{enumerate}

\section{Core Assumptions}
\label{sec:core-assumptions}

Throughout the project, some assumptions have been made. These are used to narrow the scope of the project and allow more time to be spent focusing on the core aim of the project, which is high-dimensional data. The assumptions are:
\begin{enumerate}
	\item Datasets have a ``high" number of dimensions ($\geq 10$), meaning the performance of the index structures will be measured using data with at least 10 dimensions. Data using less dimensions may be used for the purposes of understanding how the implementations behave with respect to dimensionalitym however.
	\item Datasets will be able to fit into the main memory (i.e. RAM) of the machine used for evaluation. That is, none of the data is paged to secondary memory and page accesses causing reduced performance does not need to be considered.
	\item Datasets are dynamic, meaning points may be inserted, deleted or updated at any time.
	\item Structures only store \textit{unique} points, meaning a structure cannot store two points with identical coordinates
\end{enumerate}

Furthermore, the project is focused on the \textbf{scientific visualisation} applications of multi-dimensional search, and not database indexing. Therefore, the performance of the implemented index structures will be evaluated primarily using scientific datasets with continuous domains, instead of discrete data pulled from databases.