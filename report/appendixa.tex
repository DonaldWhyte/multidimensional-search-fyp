\chapter{Personal Reflection}
\centerline{\rule{149mm}{.02in}}
\vspace{2cm}

During my time in industry last year, I worked on several projects end-to-end. Each project provided their own unique challenges, with drove me to improve both my technical and professional skills to rise the challenge and deliver. This project was no exception.

I will admit that, after returning to university with several projects under my belt, I was not expecting the final year project to be as difficult as I found it. What separated this project from the others was the heavy focus on independent research and the amount of critical thinking required to make the necessary decisions. Multi-dimensional search is a huge research area, much bigger than I had imagined. I started the project with very little knowledge on the field, so even just deciding the scope of the project required a significant amount of time and thought. Every decision made in the project required me to think carefully and objectively. Is this the right data to use? Is this the right structure to implement? The project had many ups and downs. There were times where I made wrong decisions and was burnt by it. By the end of the project, the core hypothesis driving my actions was disproven.

The main advice I'd give to students who are yet to start their final year projects, or anyone starting any kind of project, is \textbf{don't be afraid of making the wrong decisions}. One has to have the courage to make decisions, since that's the only way to acquire the required knowledge and achieve the desired outcome. I would even argue that most ``right" decisions are found by making previously wrong decisions and understanding \textit{why} they were wrong. Such is the nature of the scientific method. Not being afraid of making the wrong decisions or hypotheses allows one to maintain skepticism. Never try and fit the data to a hypothesis or have a biased interpretation. Try as hard as possible to break hypotheses, since they are only truly valuable if they have been proved correct after rigorous skepticism.

I started the implementation phase of my project with a hypothesis I wanted to prove. Early test results matched my hypothesis. When this happened, I could have simply stopped exploration there and deemed my hypothesis to be correct. However, not satisfied my evaluation was comprehensive enough, I continued to perform more and more tests and eventually found out my hypothesis turned out to be wrong. Initially, I was frustrated that I spent so much time implementing something which ended up resulting in proving the exact opposite of what I wanted to find.

What I learnt was TODO. Having empirical data to TODO 



TODO: constantly re-evaluate w -- ALWAYS THINK OF THE BIGGER PICTURES
TODO: I know you've all heard this before, but do not neglect write-up and presentation -- your work is no good if you can't communicate it to others

TODO: use supervisor and other staff as appropriate
TODO: be realistic