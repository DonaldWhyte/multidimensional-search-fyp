\chapter{Personal Reflection}
\centerline{\rule{149mm}{.02in}}
\vspace{2cm}

During my time in industry last year, I worked on several projects end-to-end. Each project provided their own unique challenges, which drove me to improve both my technical and professional skills to the level required to deliver. This project was no exception.

I will admit that, after returning to university with several projects under my belt, I was not expecting the final year project to be as difficult as I found it. What separated this project from the others was the heavy focus on independent research and the amount of critical thinking required to make the necessary decisions. Multi-dimensional search is a huge research area, much bigger than I had imagined. I started the project with very little knowledge on the field, so even just deciding the scope of the project required a significant amount of time and thought. Every decision required me to think carefully and objectively. Is this the right data to use? Is this the right structure to implement? Is this a fair and unbiased evaluation approach?

The project had many ups and downs. There were times everything seemed to go right and times I made the wrong decisions and was burnt by it. A lot of time at the start was spent planning all the work to be done throughout the three months of the project. Despite the time spent on the plan, many of this had to be changed or adapted on-the-fly to react to unexpected change. During the middle of the project, the core hypothesis driving my actions was disproved by the results of an evaluation. I had to react quickly and make sure I used the knowledge gained from the evaluation to steer the project in another direction, developing a new plan. I found this very challenging to do. Nevertheless, the project as a whole was a satisfying experience and I have definitely learnt from it.

TODO: how industry helped prepare me for project

TODO: what was it like doing this w/ no structured timetable? As a self-contained block? Just the project and nothing else???

\section{What I Have Learned}

I have learnt much from this project, but the especially valuable lessons were:
\begin{itemize}
	\item \textbf{Background Research} -- I have learnt how to dive into a field I have very little knowledge about, find relevant literature in a methodical way and synthesise that literature into higher-level insights about the field.
	\item \textbf{Risk Management} -- research projects generally involve high amounts of risks and I believe this project was no exception. This project's implementation phase in particular carried lots of risk -- a structure could take longer to implement than expected, may not perform as well as expected or I might even fail to find the necessary information to implement the structure (e.g. the Splay Quadtree). I had to become better at planning for risk and mitigating its effects to ensure the project continued running smoothly.
	\item \textbf{Methodical Evaluation} -- this is the first project that has required me to perform such an in-depth, rigorous evaluation. This experience has taught me how to methodically evaluate the deliverables, considering different evaluation approaches and how they might affect (or even bias) the final results
\end{itemize}

\section{What I Would Do Differently}

Doing this project again TODO.

\section{Advice to Fellow Students}

The main advice I'd give to students who are yet to start their final year projects, or anyone starting any kind of project, is to \textbf{not be afraid of making the wrong decisions}. One has to have the courage to make decisions, since that's the only way to achieve the desired outcome. Nearly everything worth doing involves some measure of risk and has the possibility of failure. I would even argue that most ``right" decisions are found by making previously wrong decisions and understanding \textit{why} they were wrong. Such is the nature of the scientific method.

I started the implementation phase of my project with a hypothesis I wanted to prove. Early test results matched my hypothesis. When this happened, I could have simply stopped the evaluation there and deemed my hypothesis to be correct. However, not satisfied my evaluation was comprehensive enough, I continued to perform more and more tests and eventually found out my hypothesis turned out to be wrong. Initially, I was frustrated that I spent so much time implementing something which ended up proving the exact opposite of what I wanted to find. However, I realised that these results, even if it was not what I wanted to find, are valuable. Losing the fear of making wrong decisions/hypotheses allows one to maintain skepticism. Never try and fit the data to a hypothesis or have a biased interpretation. In fact, try as hard as possible to break hypotheses; rigorous skepticism is the only way one can be truly confident in their work and defend it against criticism from others.

Another piece of advice I have is to \textbf{always think of the bigger picture}. Always remember what the project is actually trying to achieve, \textit{why} this achievement is useful and where the current task fits into a greater context of the project and its domain. It is very easy to get caught up in small technical details about specific tools or implementations which do not move the project forward. Constantly re-evaluate where the project is, what work is currently being done and if that work is actually getting the project any closer to its goal. There were a couple of times where I spend days trying to implement certain functionality before taking a step back and re-evaluating the situation to determine if what I was implementing was truly worth the time investment. This is especially important since the project's timeframe is so short.

Finally, \textbf{be realistic about the project's goals}. Ambition causes people to force themselves into situations that will push them to learn more, which is good, but the project is short. Furthermore, large chunks of this short timeframe will be spent on background research and writing the mid-project/final reports. Therefore, students are not expected to develop the latest and greatest techniques/applications. It is getting experience in the \textit{process} which is the valuable part. The process of diving into a new field, identifying where a contribution can be made, making a commitment on what to deliver and following through with that commitment, managing the inevitable changes that will occur throughout. Embracing this process will allow the project go much smoother and induce less stress.

\section{Final Remarks}

The project was challenging at times, but without those challenges it would not have been as satisfying to complete. While the end results were not as efficient as I had wanted, I hope someone will find use in the project's deliverables for their own work in the future. Above all, I greatly enjoyed this project (even if I probably spent far too many evenings in the lab). I found myself becoming highly interested in the field of multi-dimensional search, and I have only scratched the surface of it. Perhaps one day I will revisit the field and delve much deeper!