\section{Conclusion}

In this review, the multi-dimensional search problem has been defined. Sequential scan could not handle the volumes of data required and was deemed an inefficient solution, resulting in a huge collection of index structures being developed throughout the past forty years.

Each index structure has its own advantages and disadvantages because they were developed to tackle specific challenges. While some structures generally outperform others, there is no ``best" structure. When considering which to use for a given application, it is important consider some key aspects about the data and the application it is being used in (see Section \ref{sec:structure-decision}). The answers to these questions can help guide the decision on which index structure to use.

The one-dimensional search problem is generally well solved and fast, dynamic index structures capable of storing huge amounts of data exist. For a low number of dimensions (e.g. 2 or 3), research has focused more on developing simple index structures with lightweight memory requirements. For high-dimensional data, focus appears to be shifting away from tree-based structures that recursively decompose space and towards linear or hash-based structures. This is because tree-based approaches have been shown to have limited, if no, performance gain over sequential-based approaches when dealing with large numbers of dimensions \cite{md-structures-samet}. There has been research into parallelising search, but it remains a difficult problem, with little focus being given to parallelising high-dimensional search specifically.