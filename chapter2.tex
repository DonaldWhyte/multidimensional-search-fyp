\chapter{The Second Chapter}
\label{CHAP_SECOND}
\centerline{\rule{149mm}{.02in}}
\vspace{2cm}
\section{Overview}

This is my second chapter.


\section{How to Include Loads of Mathematical Symbols}
%        ********************************************

This is a presentation of the possibilities of \LaTeX\ including
equations, tables and a number of mathematical signs,
see~\cite{LaTeX:1986}.

\subsection{Differences in Equation Representation}

 The number of an equation can be set up automatically,
\begin{equation}
  Q = I_{s} + \lambda_{o} I_o + \lambda_{v} I_v 
\end{equation}
or no equation number may be produced,
\[
   Q = I_{s} + \lambda_{o} I_o + \lambda_{v} I_v
\]


\subsection{How to do lists}
%           ===============

To do lists we can do
\begin{enumerate}
  \item All triangles have areas less than a specified tolerance, or
  \item No triangles fail the test, or
  \item A preset number of triangles have been produced;
\end{enumerate}  
or (without numbers)
\begin{itemize}
  \item All triangles have areas less than a specified tolerance, or
  \item[ii)] No triangles fail the test, or
  \item[c)] A preset number of triangles have been produced;
\end{itemize}
Besides the environments \texttt{enumerate} and \texttt{itemize} there is
\texttt{description}, in which each item gets its specified label, set in
bold face:
\begin{description}
  \item[1st] This is item 1.
  \item[2nd] This is item 2.
  \item[This is a long label] followed by a long item.

    Note that items may consist of several paragraphs.
\end{description}


\subsection{Maths in text}
%           =============

For an general triangle $\triangle$ABC, let $\delta$ be the length 
of the longest side of $\triangle$ABC and let $\beta$ be the distance 
between that side and the opposite vertex.

If $\triangle$ABC is equilateral then
$\gamma_{ABC} = \frac{\delta}{\beta} = \frac{\sqrt{3}}{2}$,
so a measure of deviation is
$\left| \gamma_{ABC} - \frac{\sqrt{3}}{2} \right|$.
measure $\alpha$, where for $\triangle$ABC, 
\begin{equation}
  \alpha = \frac{4\sqrt{3} \cdot \text{~area of~} \triangle ABC}
                {AB^2 + AC^2 + BC^2} 
\end{equation}
The factor of $4\sqrt{3}$ is such that for an equilateral,
($60^\circ,60^\circ,60^\circ$), triangle, $\alpha = 1$. 

\subsubsection{Equations}
%              ---------
\begin{equation}
 \nabla^2 \xi = \xi_{xx} + \xi_{yy}   = P(x,y)
\end{equation}
\begin{equation}
 \nabla^2 \eta = \eta_{xx} + \eta_{yy}   = Q(x,y)
\end{equation}

\begin{align}
  &\text{Smoothness,} & 
  I_s &= \int_{D}  [(\nabla \xi)^{2} + (\nabla \eta)^{2} ] \dd V   \\
  &\text{Orthogonality,} &
  I_o &= \int_{D}  [(\nabla \xi \cdot \nabla \eta)^{2} ] \dd V   \\
  &\text{Weighted volume,} &
  I_v &= \int_{D}  wJ \dd V    
\end{align}
where $J = x_{\xi} y_{\eta} - x_{\eta} y_{\xi} $, is the Jacobian of
the transformation, and $w=w(x,y)$ is a given function which weights
any features of the grid which require consideration, e.g.\ clustering
to a line.
 
\begin{equation}
 \int^{x_{i+1}}_{x_i} w(x) \dd x =\text{~constant~} = \frac{1}{N}
 \int^{b}_{a} w(x) \dd x
\end{equation}

\begin{eqnarray}
  \Delta x w &=& \text{~constant} \nonumber \\
  P &=& \frac{\alpha}{J^{2}} \frac{1}{w_{1}} \PD{w_{1}}{\xi} \nonumber \\
  v_{i,j} &=& \int\limits_{\text{cell}} [ w(x,y) - w_{av} ]^2 \dd x \dd y
\end{eqnarray}
where $w_{av}$ is defined as
\begin{equation}
  w_{av} = \frac{\int_{\text{cell}} w(x,y) \dd x \dd y }
               {\int_{\text{cell}} \dd x \dd y}
\end{equation}
and the functional to be minimised is
\begin{equation}
  V = f + \gamma \sum_{i,j} v_{i,j}
\end{equation}
where $f$ is a measure of smoothness.

$\Delta z$ is defined as
\begin{equation}
  \Delta z =( \Delta x \sin\theta - \Delta y \cos\theta)(s-1)
\end{equation}

\begin{equation}
 f(x,y) = \lambda_{1} x^{2} + \lambda_{2} y^{2} + cx + dy + e
 \;\;\;\; \lambda_{1}, \lambda_{2} > 0
\end{equation}

\subsubsection{How to leave a space in text}
%              ----------------------------
Let $e = \overline{v_{i}v_{j}}$ be an internal edge of a triangulation
$T$, and let $T_{1}$ and $T_{2}$ be triangles sharing the common edge, $e$.
\vspace{3.5cm} % there is hardly any need for this
\[ P_{1}(x,y) = a_{1} x + b_{1} y + c_{1} \]
$$ P_{2}(x,y) = a_{2} x + b_{2} y + c_{2} $$
are linear interpolants of $f_{T}$ on triangles $T_{1} , T_{2}$
respectively.

\subsubsection{How to do modulus signs, norms and arrays}
%              -----------------------------------------
Let $\left| \theta \right| < \frac{\pi}{2}$. Then
\begin{align}
 s(f_{T},e) &= \left| \left| {\boldmath h} \right| \right| \\
 \intertext{where}
 {\boldmath h} &= \left( 
   \begin{array}{c}
     \left| P_{1}(x_{l},y_{l}) - F_{l} \right| \\
     \left| P_{2}(x_{k},y_{k}) - F_{k} \right|
   \end{array}
 \right)
\end{align}

\subsubsection{Mathematical symbols} 
%              --------------------
In math mode you can use Greek characters such as $\alpha$, $\beta$,
$\gamma$ or $\Lambda$, $\Xi$, $\Pi$, a special font for calligraphic
capitals, such as $\mathcal{P}$, $\mathcal{Q}$, $\mathcal{R}$ and lots
of other symbols:
\[\arraycolsep=3mm \renewcommand{\arraystretch}{1.5}
  \begin{array}{*{10}c}
    \pm & \equiv & \times & \div & \cdot & \cap & \cup & \vee & \wedge &\circ\\
    \bullet & \diamond & \lhd & \rhd & \oplus & \le & \ge & \ll & \gg & \in \\
    \subset & \supset & \subseteq & \supseteq & \ne & \approx & \sim & 
    \prec & \succ & \ni
  \end{array}
\]
Arrows and accents exist, as well as predefined names for functions like
$\log$ or $\exp$. For more symbols check \cite{symbols}.


\subsection{Tables}
%           ======

\textbf{Numerical Results}

\begin{table}[htbp]
 \begin{center}
 \begin{tabular}{||l|c|c|c|c||} \hline
         & \multicolumn{2}{c|}{$M_{\infty} = 0.675$, $10\%$ bump}
         & \multicolumn{2}{c||}{$M_{\infty} = 1.4$, $4\%$ bump} \\ \cline{2-5}
  Scheme & CFL & \# its.         & CFL & \# its.          \\ \hline
  MA     & 0.7 & 21443           & 0.7 & 5940             \\
  HESU   & 0.2 & RMS $< 10^{-7}$  & 0.2 & RMS $< 10^{-12}$ \\
  HEFSU  & 0.2 & 132818          & 0.1 & 83072            \\
  HELW   & 0.2 & RMS $< 10^{-6}$  & 0.2 & RMS $< 10^{-6}$  \\
  EHLW   & 0.2 & RMS $< 10^{-6}$  & 0.2 & RMS $< 10^{-12}$ \\ \hline
 \end{tabular}
 \end{center}
 \caption{Number of iterations for convergence to steady state for
          the two transonic circular arc bump test cases.}
 \label{chtab}
\end{table}


\subsection{Listing Program Code}
%           ====================

You can use the \texttt{verbatim}-environment
\begin{verbatim}
inline const char* DwString::c_str() const
{
    if (mRep->mRefCount > 1 && mRep != sEmptyRep) {
        DwString* xthis = (DwString*) this;
        xthis->_copy();
    }
    return &mRep->mBuffer[mStart];
}
\end{verbatim}
or use what specialised packages like \texttt{listings} provide.

