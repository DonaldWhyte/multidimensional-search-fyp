\section{Next Steps: Design and Implementation}
\label{sec:next-steps}

After the submission of this report, the first iteration of the ``Design and Implementation" phase will begin. In these two weeks, the development environment will be set up and an \textbf{evaluation framework} will be created. This framework will be a C++ program that will facilate fast evaluation by allowing multiple index structures to be tested at once (using the test process described in Section \ref{sec:evaluation}), with multiple datasets. The framework will automatically generate the evaluation measures as text files and figures, so they can easily be understand and incorporated into the report. More details on the design and features of this framework will be given in the final report.

The first iteration will also implement the evaluation baselines (sequential scan and quadtree) and a chosen index structure. Initially, the Pyramid tree was going to be implemented because the structure appears to perform very well in a high-dimensional setting (based on the literature). However, through the Visualization Toolkit (VTK)\cite{vtk}, the School already has a working implementation of the Pyramid tree. Taking the School's interests into account, it was felt that implementing the Pyramid tree again, even if the new implementation has greater performance, would not provide as much insight into high-dimensional search as implementing a structure that the School does not currently have an implementation of.

Therefore, the \textbf{splay quadtree} has been chosen as the index structure to implement. It was chosen because of its self-adjusting behaviour. From the review of literature performed, it appears self-adjusting structures have not been evaluated against high-dimensional data. By implementing the splay quadtree, insight into how these types of structures perform with higher dimensions can be gained. Furthermore, the structure has low memory overhead, making it useful for storing the large astrophysics dataset described in Section \ref{sec:evaluation} whilst keeping it in main memory (see assumption (2) in Section \ref{sec:core-assumptions}). After this iteration, the remaining iterations will attempt to optimise the splay quadtree or choose to explore a different index structure if it is shown that, algorithmically, the structure performs poorly with high-dimensional data.

Additionally, the ``Final Report Write-up"" phase will begin and run in parallel with the ``Design and Implementation" phase after the first iteration.
